\documentclass[conference]{IEEEtran}
\IEEEoverridecommandlockouts
% The preceding line is only needed to identify funding in the first footnote. If that is unneeded, please comment it out.
\usepackage{cite}
\usepackage{amsmath,amssymb,amsfonts}
\usepackage{algorithmic}
\usepackage{graphicx}
\usepackage{textcomp}
\usepackage{xcolor}
\def\BibTeX{{\rm B\kern-.05em{\sc i\kern-.025em b}\kern-.08em
    T\kern-.1667em\lower.7ex\hbox{E}\kern-.125emX}}
\begin{document}

\title{Paper Title*\\
{\footnotesize \textsuperscript{*}Note: Sub-titles are not captured in Xplore and
should not be used}
\thanks{Identify applicable funding agency here. If none, delete this.}
}

\author{\IEEEauthorblockN{1\textsuperscript{st} Given Name Surname}
\IEEEauthorblockA{\textit{dept. name of organization (of Aff.)} \\
\textit{name of organization (of Aff.)}\\
City, Country \\
email address}
\and
\IEEEauthorblockN{2\textsuperscript{nd} Given Name Surname}
\IEEEauthorblockA{\textit{dept. name of organization (of Aff.)} \\
\textit{name of organization (of Aff.)}\\
City, Country \\
email address}
\and
\IEEEauthorblockN{3\textsuperscript{rd} Given Name Surname}
\IEEEauthorblockA{\textit{dept. name of organization (of Aff.)} \\
\textit{name of organization (of Aff.)}\\
City, Country \\
email address}
\and
\IEEEauthorblockN{4\textsuperscript{th} Given Name Surname}
\IEEEauthorblockA{\textit{dept. name of organization (of Aff.)} \\
\textit{name of organization (of Aff.)}\\
City, Country \\
email address}
\and
\IEEEauthorblockN{5\textsuperscript{th} Given Name Surname}
\IEEEauthorblockA{\textit{dept. name of organization (of Aff.)} \\
\textit{name of organization (of Aff.)}\\
City, Country \\
email address}
\and
\IEEEauthorblockN{6\textsuperscript{th} Given Name Surname}
\IEEEauthorblockA{\textit{dept. name of organization (of Aff.)} \\
\textit{name of organization (of Aff.)}\\
City, Country \\
email address}
}

\maketitle

\begin{abstract}
This paper presents a method to sniff data transferred between a successful bluetooth connection without the knowledge of either of the devices. We can do this with the help of a Bluetooth hardware platform (Ubertooth in our case), a platform to perform Man-in-the-Middle attacks such as Gattacker and a Master-Slave device duo. We used a Linux system and a Raspberry Pi to fulfill those roles.

Man-in-the-Middle attacks is an attack that has got a high success rate when it comes to Bluetooth devices. But if a BLE connection has been successfully established, information leakage is improbable. So this paper discusses the sniffing data from an already established connection. We also discuss some future prospects in this subject
\end{abstract}

\section{Bluetooth}
\subsection{A Brief Overview}
Bluetooth Low Energy hit the market in 2011 as Bluetooth 4.0. The development of BLE was a major development in the field of Internet-of-Things (IoT) devices. Tough there are some overlaps with the Bluetooth Classic, it is quite different from it in terms of its functionality. While Bluetooth Classic is a comparatively faster means of data transfer continuously so ideally useful for streaming music or receiving calls in your car stereo and stuff like that. BLE differs from classic in the fact that though not as fast, it consumes considerably low power which makes it ideal for small devices which only transfer data of much smaller magnitude less periodically. This can help conserve battery, particularly for wearable devices which make use of BLE for data transfer with its environment. Thus BLE has been used vastly in IoT devices due to its considerably low power consumptions and also the fact that it "sleeps" when it is not transferring data adds to its advantages.

\subsection{Working of BLE}
Bluetooth works in the same spectrum range as Classic ie 2.4 GHz. A BLE network comprises mainly of two components: the central device, mostly a smartphone or PCs having more CPU processing power; and a peripheral device, which is usually a sensor or any other low energy device of that sort, connected to the central device. The devices advertises two types of data: Advertising packets and Scan-Response packets; of which Advertising packets are mandatory. Both the payloads are similar, both able to contain 31 bit of data but Advertising packets needs to be constantly sent to inform the central devices in range of its presence. 
 
The GAP or Generic Address Profile of a device determines the role of the particular device in the network and thus controls the advertising and connections. It makes the device visible and controls it's interaction with the world. GATT or Generic Attribute Profile controls the data transaction between the Central and Peripheral devices. GATT comes into play only after the connection is established and advertising is stopped. It will stop advertising when a connection is established ensuring only one device is connected at a time. GATT makes use of a generic data protocol called the Attribute Protocol (ATT), which is used to store Services, Characteristics and related data in a simple lookup table using 16-bit IDs for each entry in the table.
\section{Technical Background}
Here we give a background info on how BLE connections work and the possible challenges that might be encountered. Since we are focused on attacks after the connections we don't do an in-depth analysis of device advertising.
\subsection{BLE Connection}
The most important thing about Bluetooth connections is that, the connection is exclusive, that is at a time only one central device can connect to a peripheral device. So once a connection os established, it stops sending advertisements thus it remains largely invisible when it is connected. 

Upon successfully establishing a connection, communication happens both ways, ie from both the central device and the peripheral device. GATT defines the low level communication between the devices and deals with how data is transfered between the devices. It defines a hierarchy on the transferred data by organizing the data into sections called services which are in turn further divided into sets of user data called characteristics. Attribute value hold the data content of the specific attribute which can be of any data type thought the length is restricted to 512 bytes.

\subsubsection{Services}
The topmost tier which categorizes conceptually related attributes into a single category inside the GATT. As mentioned earlier services are again categorized into characteristics. They are naturally 16 bit UUID which are basically categorized into two: Primary and Secondary services. While the primary services is the standard GATT service which contain relevant useful functionalities of the GATT server, secondary services functions as a modifier having no existence on its own, thus they rarely used. A GATT can contain more than one service declarations in it.

\subsubsection{Characteristics}
The lowest level in the GATT transaction hierarchy which are the containers of user data. Similar to the services, characteristics distinguishes via 16 bit UUIDs. The characteristics have two attributes: the characteristic declaration, which contains the meta data of the user data; and the characteristic value, which contains the user data. This is the part of the GATT which concerns with the interaction  with the BLE peripheral, both to and fro.

\subsection{Data Sniffing Emulation and Capture}
Our attack takes place only after a successful BLE connection is established thus advertisements are obviously out of the picture as the device stops sending the moment a successful connection is established. Our idea is to emulate the central device and take its place instead of the original device to get control of the peripheral as well as the  thus having access to otherwise inaccessible information. For that we should first break the connection by emitting a RF using SDR which should match the base frequency of the Bluetooth chip to break the connection, emulate the device, sniff data packets from the Data transfer channels covertly.  there also lies the challenge of successfully emulating the device with their MAC addresses, specific UUIDs and characteristics. Upon successfully emulating the packets, we use a packet capture like Wireshark to capture the transfered packets and read them.

\section{Man-in-the-Middle attacks}
\subsection{Introdution}
After a connection is established there is only data transfer and broadcasting between the 2 devices along 40 channels, 2 MHz each. 37 channels are used for connection data while ch: 37, 38 and 39 are used for broadcasting. In a connection there is only one broadcaster while the other is the observer. BLE connections use a technique known as frequency hopping spread spectrum wherein the broadcasting channels are changed from time to time. But since the broadcasting is done only through 3 channels we can round up the probabilities to these three channels for sniffing useful data and it won't "hop" any further. So we will be focusing on sniffing data from these 3 channels. We make use of Ubertooth, a capable Bluetooth sniffer which also has an open source software platform and also an SDR to block connection paths along with a Raspberry Pi as an operating platform.

There are a couple of open source BLE sniffing packages and platforms to perform Man-in-the-Middle attacks like Gattacker, BLE Suite and BTLEJuice available which are all effective for our attack. The goal of the attack is to manipulate the information sent between the broadcaster and the observer, without their knowledge.

\subsection{Setup}
The main components include an Ubertooth, to capture and transmit data packets; an SDR to transmit RF of specific frequency to create interference in the connection a Linux system and also a Raspberry Pi. In this setup the Raspberry Pi becomes the Master device while the Linux system becomes the Slave device. After scanning the type of chip set being used in the IoT device we are able to find the frequency range of data trasnsmission, thus we can emit RF of the required frequency to break the connection. We then have the Ubertooth to capture the packets that are being advertised and use it to emulate the device using the Slave device. The Master device on the other hand starts sending advertisements in large magnitudes to ensure that the devices connects with the Slave device. 

\subsection{RF Transmission and Emulation}
As mentioned before a connection is established between the smart bulb and the mobile device, we scan the devices to find the hardware being used which can give us the technical details like the frequency of its transmission, base UUIDs and details like that. We then use the SDR to transmit a RF of the frequency which matches the discovered frequency of working. Thus we are able to break the already active connection thus the devices start their advertising again. This time, however, we make use of Gattacker to capture these advertised data packets and emulates device by mimicking the devices properties. Gattacker creates a platform to use captured packets from Ubertooth to emulate the device in the Raspberry Pi. It also successfully clones the MAC address of the bulb to spoof the Mobile phone, so as to match the GATT cache of the mobile device.

After the successful emulation, there are 3 devices advertising data packets at the same time. The bulb, the mobile and the emulated Linux Slave device. But there is a difference in the magnitude of the transmission between the bulb and the Slave device. Since both are trying to connect to the device, there should be a factor which would determine mobile device connect with the Slave instead of the bulb itself. Gattacker also enables us to advertise packets in much larger magnitude compared to the advertisements of the bulb. This is because, bulbs transmit advertisements in lesser magnitudes to minimize power consumption and thus the large amount of advertisements almost guarantees the establishment of a successful connecton beteeen the slave device and the bulb.

\subsection{Packet Capture}
\subsubsection{Important parameters}

\paragraph{Bluetooth Address}
A unique 48 bit address that is commonly known as \verb|BD_ADDR|. The first 24 bits identifies the manufacturer while the lower 24 bits are unique to that address. It is quite similar to the MAC Address in IP connections

\paragraph{\textbf{Access address}}
Upon the establishment of a successful connection, we have access to the data that are being transferred between the connected devices. As mentioned before there are 40 channels of data transmisson, where 3 channels ae used for broadcasting. The frequency hopping spread spectrum makes interference difficult but not impossible. Here, they use a parameter called the Access Address which ensures that the data channels have negligible chances of collision  while data transfer occurs. In fact, this address acts as the identification for a connection. It is 4 bytes long which is used for the master device access address identification. 

\paragraph{\textbf{Index}}
The channel of transmission i.e. the frequency index of the channel used.

\paragraph{\textbf{Protocol Data Unit or PDU}}
The PDU contains info on the data that is being transferred that is the type of data transferred, the \verb|SCAN_REQ| a request given by the scanner device to start transmission and \verb|SCAN_RES|, a response given by the Central device upn recieving the \verb|SCAN_REQ|.

\subsection{Abbreviations and Acronyms}\label{AA}
Define abbreviations and acronyms the first time they are used in the text, 
even after they have been defined in the abstract. Abbreviations such as 
IEEE, SI, MKS, CGS, ac, dc, and rms do not have to be defined. Do not use 
abbreviations in the title or heads unless they are unavoidable.

\subsection{Units}
\begin{itemize}
\item Use either SI (MKS) or CGS as primary units. (SI units are encouraged.) English units may be used as secondary units (in parentheses). An exception would be the use of English units as identifiers in trade, such as ``3.5-inch disk drive''.
\item Avoid combining SI and CGS units, such as current in amperes and magnetic field in oersteds. This often leads to confusion because equations do not balance dimensionally. If you must use mixed units, clearly state the units for each quantity that you use in an equation.
\item Do not mix complete spellings and abbreviations of units: ``Wb/m\textsuperscript{2}'' or ``webers per square meter'', not ``webers/m\textsuperscript{2}''. Spell out units when they appear in text: ``. . . a few henries'', not ``. . . a few H''.
\item Use a zero before decimal points: ``0.25'', not ``.25''. Use ``cm\textsuperscript{3}'', not ``cc''.)d
\end{itemize}

\subsection{Equations}
Number equations consecutively. To make your 
equations more compact, you may use the solidus (~/~), the exp function, or 
appropriate exponents. Italicize Roman symbols for quantities and variables, 
but not Greek symbols. Use a long dash rather than a hyphen for a minus 
sign. Punctuate equations with commas or periods when they are part of a 
sentence, as in:
\begin{equation}
a+b=\gamma\label{eq}
\end{equation}

Be sure that the 
symbols in your equation have been defined before or immediately following 
the equation. Use ``\eqref{eq}'', not ``Eq.~\eqref{eq}'' or ``equation \eqref{eq}'', except at 
the beginning of a sentence: ``Equation \eqref{eq} is . . .''

\subsection{\LaTeX-Specific Advice}

Please use ``soft'' (e.g., \verb|\eqref{Eq}|) cross references instead
of ``hard'' references (e.g., \verb|(1)|). That will make it possible
to combine sections, add equations, or change the order of figures or
citations without having to go through the file line by line.

Please don't use the \verb|{eqnarray}| equation environment. Use
\verb|{align}| or \verb|{IEEEeqnarray}| instead. The \verb|{eqnarray}|
environment leaves unsightly spaces around relation symbols.

Please note that the \verb|{subequations}| environment in {\LaTeX}
will increment the main equation counter even when there are no
equation numbers displayed. If you forget that, you might write an
article in which the equation numbers skip from (17) to (20), causing
the copy editors to wonder if you've discovered a new method of
counting.

{\BibTeX} does not work by magic. It doesn't get the bibliographic
data from thin air but from .bib files. If you use {\BibTeX} to produce a
bibliography you must send the .bib files. 

{\LaTeX} can't read your mind. If you assign the same label to a
subsubsection and a table, you might find that Table I has been cross
referenced as Table IV-B3. 

{\LaTeX} does not have precognitive abilities. If you put a
\verb|\label| command before the command that updates the counter it's
supposed to be using, the label will pick up the last counter to be
cross referenced instead. In particular, a \verb|\label| command
should not go before the caption of a figure or a table.

Do not use \verb|\nonumber| inside the \verb|{array}| environment. It
will not stop equation numbers inside \verb|{array}| (there won't be
any anyway) and it might stop a wanted equation number in the
surrounding equation.

\subsection{Some Common Mistakes}\label{SCM}
\begin{itemize}
\item The word ``data'' is plural, not singular.
\item The subscript for the permeability of vacuum $\mu_{0}$, and other common scientific constants, is zero with subscript formatting, not a lowercase letter ``o''.
\item In American English, commas, semicolons, periods, question and exclamation marks are located within quotation marks only when a complete thought or name is cited, such as a title or full quotation. When quotation marks are used, instead of a bold or italic typeface, to highlight a word or phrase, punctuation should appear outside of the quotation marks. A parenthetical phrase or statement at the end of a sentence is punctuated outside of the closing parenthesis (like this). (A parenthetical sentence is punctuated within the parentheses.)
\item A graph within a graph is an ``inset'', not an ``insert''. The word alternatively is preferred to the word ``alternately'' (unless you really mean something that alternates).
\item Do not use the word ``essentially'' to mean ``approximately'' or ``effectively''.
\item In your paper title, if the words ``that uses'' can accurately replace the word ``using'', capitalize the ``u''; if not, keep using lower-cased.
\item Be aware of the different meanings of the homophones ``affect'' and ``effect'', ``complement'' and ``compliment'', ``discreet'' and ``discrete'', ``principal'' and ``principle''.
\item Do not confuse ``imply'' and ``infer''.
\item The prefix ``non'' is not a word; it should be joined to the word it modifies, usually without a hyphen.
\item There is no period after the ``et'' in the Latin abbreviation ``et al.''.
\item The abbreviation ``i.e.'' means ``that is'', and the abbreviation ``e.g.'' means ``for example''.
\end{itemize}
An excellent style manual for science writers is \cite{b7}.

\subsection{Authors and Affiliations}
\textbf{The class file is designed for, but not limited to, six authors.} A 
minimum of one author is required for all conference articles. Author names 
should be listed starting from left to right and then moving down to the 
next line. This is the author sequence that will be used in future citations 
and by indexing services. Names should not be listed in columns nor group by 
affiliation. Please keep your affiliations as succinct as possible (for 
example, do not differentiate among departments of the same organization).

\subsection{Identify the Headings}
Headings, or heads, are organizational devices that guide the reader through 
your paper. There are two types: component heads and text heads.

Component heads identify the different components of your paper and are not 
topically subordinate to each other. Examples include Acknowledgments and 
References and, for these, the correct style to use is ``Heading 5''. Use 
``figure caption'' for your Figure captions, and ``table head'' for your 
table title. Run-in heads, such as ``Abstract'', will require you to apply a 
style (in this case, italic) in addition to the style provided by the drop 
down menu to differentiate the head from the text.

Text heads organize the topics on a relational, hierarchical basis. For 
example, the paper title is the primary text head because all subsequent 
material relates and elaborates on this one topic. If there are two or more 
sub-topics, the next level head (uppercase Roman numerals) should be used 
and, conversely, if there are not at least two sub-topics, then no subheads 
should be introduced.

\subsection{Figures and Tables}
\paragraph{Positioning Figures and Tables} Place figures and tables at the top and 
bottom of columns. Avoid placing them in the middle of columns. Large 
figures and tables may span across both columns. Figure captions should be 
below the figures; table heads should appear above the tables. Insert 
figures and tables after they are cited in the text. Use the abbreviation 
``Fig.~\ref{fig}'', even at the beginning of a sentence.

\begin{table}[htbp]
\caption{Table Type Styles}
\begin{center}
\begin{tabular}{|c|c|c|c|}
\hline
\textbf{Table}&\multicolumn{3}{|c|}{\textbf{Table Column Head}} \\
\cline{2-4} 
\textbf{Head} & \textbf{\textit{Table column subhead}}& \textbf{\textit{Subhead}}& \textbf{\textit{Subhead}} \\
\hline
copy& More table copy$^{\mathrm{a}}$& &  \\
\hline
\multicolumn{4}{l}{$^{\mathrm{a}}$Sample of a Table footnote.}
\end{tabular}
\label{tab1}
\end{center}
\end{table}

\begin{figure}[htbp]
\centerline{\includegraphics{fig1.png}}
\caption{Example of a figure caption.}
\label{fig}
\end{figure}

Figure Labels: Use 8 point Times New Roman for Figure labels. Use words 
rather than symbols or abbreviations when writing Figure axis labels to 
avoid confusing the reader. As an example, write the quantity 
``Magnetization'', or ``Magnetization, M'', not just ``M''. If including 
units in the label, present them within parentheses. Do not label axes only 
with units. In the example, write ``Magnetization (A/m)'' or ``Magnetization 
\{A[m(1)]\}'', not just ``A/m''. Do not label axes with a ratio of 
quantities and units. For example, write ``Temperature (K)'', not 
``Temperature/K''.

\section*{Acknowledgment}

The preferred spelling of the word ``acknowledgment'' in America is without 
an ``e'' after the ``g''. Avoid the stilted expression ``one of us (R. B. 
G.) thanks $\ldots$''. Instead, try ``R. B. G. thanks$\ldots$''. Put sponsor 
acknowledgments in the unnumbered footnote on the first page.

\section*{References}

Please number citations consecutively within brackets \cite{b1}. The 
sentence punctuation follows the bracket \cite{b2}. Refer simply to the reference 
number, as in \cite{b3}---do not use ``Ref. \cite{b3}'' or ``reference \cite{b3}'' except at 
the beginning of a sentence: ``Reference \cite{b3} was the first $\ldots$''

Number footnotes separately in superscripts. Place the actual footnote at 
the bottom of the column in which it was cited. Do not put footnotes in the 
abstract or reference list. Use letters for table footnotes.

Unless there are six authors or more give all authors' names; do not use 
``et al.''. Papers that have not been published, even if they have been 
submitted for publication, should be cited as ``unpublished'' \cite{b4}. Papers 
that have been accepted for publication should be cited as ``in press'' \cite{b5}. 
Capitalize only the first word in a paper title, except for proper nouns and 
element symbols.

For papers published in translation journals, please give the English 
citation first, followed by the original foreign-language citation \cite{b6}.

\begin{thebibliography}{00}
\bibitem{b1} G. Eason, B. Noble, and I. N. Sneddon, ``On certain integrals of Lipschitz-Hankel type involving products of Bessel functions,'' Phil. Trans. Roy. Soc. London, vol. A247, pp. 529--551, April 1955.
\bibitem{b2} J. Clerk Maxwell, A Treatise on Electricity and Magnetism, 3rd ed., vol. 2. Oxford: Clarendon, 1892, pp.68--73.
\bibitem{b3} I. S. Jacobs and C. P. Bean, ``Fine particles, thin films and exchange anisotropy,'' in Magnetism, vol. III, G. T. Rado and H. Suhl, Eds. New York: Academic, 1963, pp. 271--350.
\bibitem{b4} K. Elissa, ``Title of paper if known,'' unpublished.
\bibitem{b5} R. Nicole, ``Title of paper with only first word capitalized,'' J. Name Stand. Abbrev., in press.
\bibitem{b6} Y. Yorozu, M. Hirano, K. Oka, and Y. Tagawa, ``Electron spectroscopy studies on magneto-optical media and plastic substrate interface,'' IEEE Transl. J. Magn. Japan, vol. 2, pp. 740--741, August 1987 [Digests 9th Annual Conf. Magnetics Japan, p. 301, 1982].
\bibitem{b7} M. Young, The Technical Writer's Handbook. Mill Valley, CA: University Science, 1989.
\end{thebibliography}
\vspace{12pt}
\color{red}
IEEE conference templates contain guidance text for composing and formatting conference papers. Please ensure that all template text is removed from your conference paper prior to submission to the conference. Failure to remove the template text from your paper may result in your paper not being published.

\end{document}
